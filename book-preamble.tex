%%%%%%%%%%%%%%%%%%
% 标题左对齐
%\usepackage[raggedright]{titlesec}
% 页边距设置
% marginpar=2cm解决margin note在ctex book下无法留出足够空白问题
% 标准的16开,适合于计算机类图书出版
\usepackage[paperheight=260mm, paperwidth=185mm, marginpar=2cm,top=3cm,bottom=3cm,left=2.54cm,right=2.54cm]{geometry}

% A4版本,适合打印
%\usepackage[paperheight=297mm, paperwidth=210mm, marginpar=2cm,top=1.65cm,bottom=1.65cm,left=2cm,right=2cm]{geometry}

% 在每一页显示4个切割标志
% FIXME使用A4打印出版,需要根据实际打印纸张尺寸设定
\usepackage[cam,a4,center]{crop}

% 每章显示名言警句
\usepackage{epigraph}

% 确定使用了tikz绘制的图片并且在图片中包含beamer中的overlay设置才需要此宏包
%\usepackage{beamerarticle}
% 页眉页脚设置
\usepackage{fancyhdr}
%\usepackage{calc}
\pagestyle{fancy} % 默认的效果也可以接受
\fancyhf{}                     % 清空页眉页脚
%页眉长度也包括边注
%TODO 页眉是否包含边注,取决于是否大量使用边注。如果没有使用边注可注释此行
%FIXME 存在两个问题:
% 1, 如果页眉包括边注,则页边章序号导航会遮盖部分页眉
% 2, 导入calc宏包似乎会导致assoccnt宏包失效,即页边章序号导航无法产生争取的章序号
%\fancyheadoffset[RO,LE]{\marginparsep+\marginparwidth}
\fancyhead[LE,RO]{\thepage}    % 页码:偶数页左,奇数页右
\fancyhead[RE]{\leftmark}      % 偶数页右
\fancyhead[LO]{\rightmark}     % 奇数页左
\fancypagestyle{plain}{ % 重新定义plain样式,在章等的首页使用plain样式
    \fancyhf{}
    \cfoot{\thepage} % 页脚中间显示页码
    \renewcommand{\headrulewidth}{0pt} % 清除页眉线
}

% 重新设置fancyhdr的headheight,避免报告Package Fancyhdr Warning: \headheight is too small (12.0pt)
\setlength{\headheight}{14pt} 

% 目录样式:解决目录中的省略号太稀疏问题
%\usepackage{tocloft} %暂时不需要这么重量级的宏包
\renewcommand\@dotsep{1} %默认值是4.5

% 图片样式:caption字体小一号,参见caption package guide
\usepackage[margin=10pt,font=small,labelfont=bf,labelsep=endash]{caption}

% 表格样式
% caption在表格上面
% 表格字体小一号
\usepackage{floatrow}
\floatsetup[table]{font=small,capposition=top}

% 章的序号样式
\ctexset{
    chapter/number = \arabic{chapter},
    chapter/numberformat = \color{blue}\zihao{0}\itshape,
}
% 调整字间距,I don't know the effect,也许ctex已经这样设置了,需要调查?
%\renewcommand{\CJKglue}{\hskip 0.9bp plus 0.03\baselineskip minus 0.03\baselineskip}
\usepackage{listings}
\usepackage{color}
\definecolor{colBg}{rgb}{1,1,1} % 白色,便于打印输出,不会和code remarks冲突
\definecolor{colKeys}{rgb}{0,0,1}
\definecolor{colIdentifier}{rgb}{0,0,0}
\definecolor{colComments}{rgb}{0.06,0.05,0.03} % 重色便于打印输出,红色打印出来会很模糊
\definecolor{colString}{rgb}{0,0.5,0}
\lstset{%
    language=Java,%可以在使用lstlisting环境时修改语言的设置
    float=thbp,%防止代码被分页打断
    basicstyle=\small\ttfamily,%
    identifierstyle=\color{colIdentifier},%
    keywordstyle=\bfseries\color{colKeys},%关键字加粗便于黑白打印输出
    stringstyle=\color{colString},%
    commentstyle=\itshape\color{colComments},%
    columns=fixed,
    tabsize=4,%
    frame=tb,% 顶部和底部加横线
    %frame=shadowbox,
    framerule=1pt,
    showspaces=false,%
    showstringspaces=false,%不显示代码字符串中间的空格标记
    %framexleftmargin=2em, % 行号包含在代码区域内
    %numbers=left,%左侧显示行号
    numberstyle=\tiny\ttfamily,%
    numbersep=1em,%
    breaklines=true,% 对过长的代码自动换行
    breakindent=10pt,%
    backgroundcolor=\color{colBg},%
    breakautoindent=true,%
    %escapebegin=\begin{CJK*}{GBK}{hei},escapeend=\end{CJK*},
    aboveskip=1em, %代码块边框
    captionpos=t,%
    %% added by http://bbs.ctex.org/viewthread.php?tid=53451
    %fontadjust,
    xleftmargin=1em, xrightmargin=\fboxsep,%设定listing左右的空白
    %texcl=true,
    % 设定中文冲突,断行,列模式,数学环境输入,listing数字的样式
    extendedchars=false,columns=flexible,
    %mathescape=true, % 启用后貌似没法在代码中使用特殊字符,比如$
    %可以在(*@和@*)之间插入LaTeX命令
    escapeinside={(*@}{@*)},
    escapechar=|
}

\AtBeginDocument{\renewcommand\lstlistingname{代码清单}}
\AtBeginDocument{\renewcommand\tablename{表}}
\AtBeginDocument{\renewcommand\figurename{图}}
\AtBeginDocument{\renewcommand\listfigurename{图~目~录}}
\AtBeginDocument{\renewcommand\listtablename{表~目~录}}
\AtBeginDocument{\providecommand\sectionname{节}}
\AtBeginDocument{\providecommand\exercisename{练习}}
\AtBeginDocument{\providecommand\examplename{例}}
\AtBeginDocument{\providecommand\solutionname{解答}}
\AtBeginDocument{\renewcommand\appendixname{附录}}
\AtBeginDocument{\renewcommand\partname{部分}}
\AtBeginDocument{\renewcommand*{\lstlistlistingname}{示例代码列表}}

\usepackage{varioref} % 为了防止refxxx没有定义错误
\AtBeginDocument{\renewcommand\reftextfaraway[1]{[在第~\pageref{#1}~页]}}
\AtBeginDocument{\renewcommand\reftextbefore{[在上一页]}}
\AtBeginDocument{\renewcommand\reftextafter{[在下一页]}}
\AtBeginDocument{\renewcommand\reftextcurrent{[在本页]}}
\AtBeginDocument{\renewcommand\reftextfaceafter{[在对页]}}
\AtBeginDocument{\renewcommand\reftextfacebefore{[在对页]}}

% 在页边显示章节序号
\usepackage[contents={},opacity=1,scale=1,color=white]{background}
\usepackage{tikzpagenodes}
\usepackage{totcount}
\usepackage{shorttoc}
\usetikzlibrary{calc}
\usepackage{assoccnt}

\newif\ifMaterial
\definecolor{light-blue}{rgb}{0.8,0.85,1}

\newlength\LabelSize
\setlength\LabelSize{2.5cm}

% auxiliary counter
\newcounter{chapshift}
\setcounter{chapshift}{-1}
\DeclareAssociatedCounters{chapter}{chapshift}

\AtBeginDocument{%
    \regtotcounter{chapter}
    \ifnum\totvalue{chapter}>0\relax
        \setlength\LabelSize{\dimexpr\textheight/\totvalue{chapter}\relax}
        \ifdim\LabelSize>2.5cm\relax
            \global\setlength\LabelSize{2.5cm}
        \fi
    \fi
}

\newcommand\AddLabels{%
\Materialtrue%
\AddEverypageHook{%
\ifMaterial%
\ifodd\value{page} %
 \backgroundsetup{
  angle=90,
  position={current page.east|-current page text area.north  east},
  vshift=12pt,
  hshift=-\thechapshift*\LabelSize,
  contents={%
  \tikz\node[fill=light-blue,anchor=west,text width=\LabelSize,
    align=center,text height=15pt,text depth=13pt,font=\large\sffamily] {\thechapter};
  }%
 }
 \else
 \backgroundsetup{
  angle=90,
  position={current page.west|-current page text area.north west},
  vshift=-12pt,
  hshift=-\thechapshift*\LabelSize,
  contents={%
  \tikz\node[fill=light-blue,anchor=west,text width=\LabelSize,
    align=center,text height=15pt,text depth=13pt,font=\large\sffamily] {\rotatebox{180}{\thechapter}};
  }%
 }
 \fi
 \BgMaterial%
\else\relax\fi}%
}

\newcommand\RemoveLabels{\Materialfalse}


% 在页边显示一个代码索引序号
\newcommand{\srcidx}[1]{%
    \makebox[0pt][l]{%
        \makebox[\linewidth+\marginparsep][l]{}%
            \fbox{#1}%
    }%
}

