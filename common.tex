% 解决了在artice/book中包含tikz图片问题:可以使用beamer的overlay,
% 参见:https://tex.stackexchange.com/questions/201114/how-can-i-include-something-with-overlay-specifications-in-a-non-beamer-class-do
%\usepackage{beamerarticle}

%\usepackage{xcolor}
%\usepackage{tikz}
% 目录样式:解决目录中的省略号太稀疏问题
%\usepackage{tocloft} %暂时不需要这么重量级的宏包
\renewcommand\@dotsep{1} %默认值是4.5

% 图片样式:caption字体小一号,参见caption package guide
\usepackage[margin=10pt,font=small,labelfont=bf,labelsep=endash]{caption}
\captionsetup[table]{position=above}
\captionsetup[longtable]{position=top}

% 表格样式
% caption在表格上面
% 表格字体小一号
% 需要分别定义table、longtable(figure)的属性
%\usepackage{floatrow}
%\floatsetup[table]{font=small,capposition=top}
%\floatsetup[longtable]{font=small,capposition=top}
% floatrow破坏了longtable水平中央对齐,在这里恢复,但是不知道有没有其他危害
% 参考:https://tex.stackexchange.com/questions/320592/floatrow-package-conflicting-with-centering-longtable
%\setlength\LTleft{\fill}
%\setlength\LTright{\fill}
%\DeclareMarginSet{centering}{\setfloatmargins{\hfill}{\hfill} \LTleft=\fill \LTright=\fill}

% 暂时解决lualatex的兼容性问题,参见:https://github.com/CTeX-org/ctex-kit/issues/283
\newdimen\cht
\newdimen\cdp

% 脚注的样式修改为带圈数字
% 参见<LaTeX入门>P119
\usepackage{pifont}
\renewcommand\thefootnote{\ding{\numexpr171+\value{footnote}}}
